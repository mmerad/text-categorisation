\section{Épuration des données}
Mettre une intro ici
\subsection{Ensemble de tests}
Premièrement, nous devions, à partir des données, créer deux ensemble : Un ensemble d'entraînement et un ensemble de test. Il est important d'évaluer la catégorisation trouvée sur un ensemble différent de celui d'apprentissage pour ...

Trois séparations  nous étaient proposé avec le corpus de textes: Lewissplit non modifié, ModApteSplit, et le ModHayes Split.
Comme indiqué dans le fichier \textsc{README}, nous avons préféré leModApteSplit (The Modified Apte), qui semble meilleur pour la catégorisation sur les balises \textit{TOPICS}.

Après cette séparation, nous obtenons les ensembles suivants : 
\begin{itemize}
\item Ensemble d'entrainement (9,603 documents): LEWISSPLIT="TRAIN";  TOPICS="YES"
\item Ensemble de test (3,299 documents): LEWISSPLIT="TEST"; TOPICS="YES"
\item Non-utilisés (8,676 documentss):   LEWISSPLIT="NOT-USED"; TOPICS="YES" or TOPICS="NO"  or TOPICS="BYPASS"
\end{itemize}
