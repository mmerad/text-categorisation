\section*{Conclusion}
\addcontentsline{toc}{section}{Conclusion}
\markboth{Conclusion}{CONCLUSION}

Il existe de nombreux algorithmes de catégorisation de texte. Notre choix s'est porté sur l'algorithme Naive Bayes et l'arbre de décision car ils sont des approches différentes. Ainsi nous avons pu tester les performance de deux classifiers différents.\\
En fin de compte nous obtenons de meilleurs résultats avec les arbres de décisions même si ces derniers souffrent d'un gros problème de performance, les empêchant notamment de pouvoir travailler sur le corpus complet.\\
Avec l'algorithme Naives Bayes nous avons obtenu des résultats un peu décevant vis à vis de nos attentes et contraire à nos prévisions concernant les titres et l'apprentissage continu.\\
Nous avons cependant perdu énormément de temps sur le formatage et la lecture des données. En effet ces dernières sont présentées dans des fichiers XML-like qui ne respectent ni les normes XML, ni le "README.txt" présent avec le corpus. 
Ce projet fut malgré tout très enrichissant et nous a permis de mieux comprendre la catégorisation de texte et le fonctionnement de deux algorithmes assez répandus. \\
\newpage

\section*{Bibliographie}
\addcontentsline{toc}{section}{Bibliographie}

\begin{thebibliography}{99}

%Un élément, la numérotation est automatique
%\bibitem{nom_du_marqueur_citation} ce nom est é mettre dans une balise \cite{nom}~
%ecrire le mot accolé au ~ pour que ca marche
%Sie
%exemple
%\bibitem{traiteRV}
%\emph{} \textsc{}
%\bibitem{These}
%\emph{Tony Doat} \textsc{Thèse intitulée : Espaces Virtuels pour l'éducation et l'illustration scientifiques} 

\end{thebibliography}
