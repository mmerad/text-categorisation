\subsection{Naives Bayes}


\paragraphs{Phase d'entraînement}  Nous récupérons un à un les textes prévus pour la phase d'entraînement. Ils seront tous traité individuellement dans un premier temps. Pour chaque texte,nous avons calculé l'espérance d'un mot dans un texte défini par un certain topic. Pour cela la fréquence de chaque mot est calculé selon le texte puis elle est divisé par la taille du texte. Cette valeur est ensuite mémorisé avec le topic associé au texte. Si le topic pour ce mot existe déjà, la moyenne est enregistré avec une pondération de 1,25 pour donner plus de poids face aux autres topics. A la fin de cette phase, nous obtenons pour chaque mot, un ensemble d'association entre topic et probabilité. 


\paragraphs{Phase de test}  Nous comparons chaque mot qui se situe dans le texte à tester avec les mots de la base que nous avons créé. Pour chaque mot, on enregistre dans une map, les topics trouvés, si le topic existe déjà, on incrémente sa fréquence et sinon on l'ajoute dans la map avec une valeur initial de 1. Une fois que tout le texte a été analysé, on extrait les deux topics qui ont les fréquences le plus élevées.
